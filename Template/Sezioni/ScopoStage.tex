%----------------------------------------------------------------------------------------
%	STAGE DESCRIPTION
%----------------------------------------------------------------------------------------
\section*{Scopo dello stage}
% Personalizzare inserendo lo scopo dello stage, cioè una breve descrizione
Lo scopo di questo stage è la realizzazione della componente "Hydrogen Custom Logon" per il sistema operativo \textit{Apple macOS}.\newline
La base di questo progetto è l'applicazione "Hydrogen" che permette il self-reset delle password dimenticate, o scadute, da parte degli utenti incrementando la sicurezza e riducendo i costi per il servizio di helpdesk.
Tale applicazione, si integra completamente con l'interfaccia di accesso \textit{Windows} (fase di richiesta dei dati di login) aggiungendovi la funzionalità di reset password. E' inoltre disponibile una versione di "Hydrogen" anche per \textit{Ubuntu Linux}.
\newline
Lo studente, mediante collaborazione con il team di sviluppo, si occuperà di realizzare i seguenti prodotti software:
\begin{itemize}
	\item Componente "Hydrogen Custom Logon" per \textit{Apple macOS}, con i seguenti obiettivi:
	\begin{itemize}
		\item Visualizzare l'icona di "Hydrogen" nella schermata di logon del sistema operativo;
		\item Visualizzare la User Interface di "Hydrogen" all'interno di un browser opportunamente limitato.
	\end{itemize}
	\item Installer per il componente realizzato, con le seguenti funzionalità:
	\begin{itemize}
		\item Installare/disinstallare la componente;
		\item Possibilità di effettuare l'installazione in modalità silente;
		\item Configurare la componente.
	\end{itemize}
\end{itemize}

