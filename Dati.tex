%----------------------------------------------------------------------------------------
%   USEFUL COMMANDS
%----------------------------------------------------------------------------------------

\newcommand{\dipartimento}{Dipartimento di Matematica ``Tullio Levi-Civita''}

%----------------------------------------------------------------------------------------
% 	USER DATA
%----------------------------------------------------------------------------------------

% Data di approvazione del piano da parte del tutor interno; nel formato GG Mese AAAA
% compilare inserendo al posto di GG 2 cifre per il giorno, e al posto di 
% AAAA 4 cifre per l'anno
\newcommand{\dataApprovazione}{Data}

% Dati dello Studente
\newcommand{\nomeStudente}{Pardeep}
\newcommand{\cognomeStudente}{Singh}
\newcommand{\matricolaStudente}{1143264}
\newcommand{\emailStudente}{pardeep.singh@studenti.unipd.it}
\newcommand{\telStudente}{+ 39 388 82 77 487}

% Dati del Tutor Aziendale
\newcommand{\nomeTutorAziendale}{Claudio}
\newcommand{\cognomeTutorAziendale}{Guarisco}
\newcommand{\emailTutorAziendale}{Claudio.Guarisco@netcom.it}
\newcommand{\telTutorAziendale}{+39 049 88 09 910}
\newcommand{\ruoloTutorAziendale}{Developer}

% Dati dell'Azienda
\newcommand{\ragioneSocAzienda}{NETCOM s.r.l.}
\newcommand{\indirizzoAzienda}{Via Fusinato 42, 35137, Padova, Italy}
\newcommand{\sitoAzienda}{www.netcom.it}
\newcommand{\emailAzienda}{info@netcom.it}
\newcommand{\partitaIVAAzienda}{P.IVA 03447430285}

% Dati del Tutor Interno (Docente)
\newcommand{\titoloTutorInterno}{Prof.}
\newcommand{\nomeTutorInterno}{Claudio Enrico}
\newcommand{\cognomeTutorInterno}{Palazzi}

\newcommand{\prospettoSettimanale}{
     % Personalizzare indicando in lista, i vari task settimana per settimana
     % sostituire a XX il totale ore della settimana
    \begin{itemize}
        \item \textbf{Prima Settimana (XX ore)}
        \begin{itemize}
            \item Introduzione alle modalità di lavoro nel team di sviluppo;
            \item Formazione sul sistema aziendale di infrastruttura virtuale;
            \item Introduzione al progetto Hydrogen;
            \item Analisi Hydrogen Custom Logon for Windows già in essere;
            \item Analisi problematica versioni diverse di MacOS;
        \end{itemize}
        \item \textbf{Seconda Settimana - Sottotitolo (XX ore)} 
        \begin{itemize}
            \item Analisi problematiche chiusura di MacOS;
            \item Identificazione requisiti;
            \item Documentazione;
        \end{itemize}
        \item \textbf{Terza Settimana - Sottotitolo (XX ore)} 
        \begin{itemize}
            \item ;
        \end{itemize}
        \item \textbf{Quarta Settimana - Sottotitolo (XX ore)} 
        \begin{itemize}
            \item ;
        \end{itemize}
        \item \textbf{Quinta Settimana - Sottotitolo (XX ore)} 
        \begin{itemize}
            \item ;
        \end{itemize}
        \item \textbf{Sesta Settimana - Sottotitolo (XX ore)} 
        \begin{itemize}
            \item ;
        \end{itemize}
        \item \textbf{Settima Settimana - Sottotitolo (XX ore)} 
        \begin{itemize}
            \item ;
        \end{itemize}
        \item \textbf{Ottava Settimana - Conclusione (XX ore)} 
        \begin{itemize}
            \item ;
        \end{itemize}
    \end{itemize}
}

% Indicare il totale complessivo (deve essere compreso tra le 300 e le 320 ore)
\newcommand{\totaleOre}{320}

\newcommand{\obiettiviObbligatori}{
	 \item \underline{\textit{O01}}: primo obiettivo;
	 \item \underline{\textit{O02}}: secondo obiettivo;
	 \item \underline{\textit{O03}}: terzo obiettivo;
	 
}

\newcommand{\obiettiviDesiderabili}{
	 \item \underline{\textit{D01}}: primo obiettivo;
	 \item \underline{\textit{D02}}: secondo obiettivo;
}

\newcommand{\obiettiviFacoltativi}{
	 \item \underline{\textit{F01}}: primo obiettivo;
	 \item \underline{\textit{F02}}: secondo obiettivo;
	 \item \underline{\textit{F03}}: terzo obiettivo;
}